\documentclass[12pt,letterpaper,noanswers]{exam}
\usepackage[usenames,dvipsnames,svgnames,table]{xcolor}
\usepackage[margin=0.9in]{geometry}
\renewcommand{\familydefault}{\sfdefault}
\usepackage{multicol}
\pagestyle{head}
\header{AM 108 Class 17}{}{Poincar\'e map}
\runningheadrule
\headrule
\usepackage{graphicx} % more modern
\usepackage{amsmath} 
\usepackage{amssymb} 
\usepackage{hyperref}
\usepackage{tcolorbox}

\begin{document}
 \pdfpageheight 11in 
  \pdfpagewidth 8.5in

\noindent 




\begin{itemize}

   
    \item There is not a class meeting on Monday.  I will be available on Slack/APMTH 108 Zoom from 1pm-2:30pm on Monday to answer questions that arise as you take the quiz.
        \item There is a quiz on Monday October 19th.  
    \item There will be a pre-class assignment for Wednesday.
    \item There will be a skill check in class on Wednesday.  The problem info is below.
 \item Problem set 07 will be due Friday October 23rd.
\end{itemize}

\hrule
\vspace{0.2cm}



\noindent\textbf{Teams}

\begin{multicols}{2}
1. 

\end{multicols}

\noindent \textbf{Teams 3 and 4}: Post screenshots of your work to the course Google Drive today.  Include words, labels, and other short notes that might make those solutions useful to you or your classmates.  Find the link in Canvas (or here: \url{https://drive.google.com/drive/u/0/folders/1GcpwvKHD4tMecpFQ4lNxN_r5Ylj7YHbd})

\vspace{0.2cm}

\hrule
\vspace{0.2cm}

\noindent\textbf{Polls}


\vspace{0.2cm}

\hrule
\vspace{0.2cm}

\noindent\textbf{Big picture}

We have been working on how to construct a phase portrait for a 2d system in $\mathbb{R}^2.$ Recently we have been trying to determine whether or where a phase portrait might have closed trajectories.

Today we are looking at a method for approximating the location of a closed trajectory and determining its stability (when we suspect such a trajectory exists).


\vspace{0.2cm}
\hrule
\vspace{0.2cm}


\noindent \textbf{Extra vocabulary / extra facts:}
\begin{tcolorbox}
For a dynamical system in $\mathbb{R}^2$ \textbf{Poincar\'e map} is a map from a curve, $S$, back to the curve $S$.  The mapping is obtained by following a trajectory from one intersection of $S$ to its next intersection of $S$.

A fixed point of a Poincar\'e map corresponds to a closed orbit of the original system.

We can use the Poincar\'e map to identify the \textbf{stability} of a limit cycle. 

\end{tcolorbox}

\noindent\textbf{Recall}
\begin{tcolorbox}

A \textbf{map} is a type of dynamical system in which time is discrete.  For example, $x_{n+1} = \cos(x_n)$ is a map.  $n$ takes on positive integer values.  Rather than finding a solution $x(t)$ that is defined for all $t\in\mathbb{R}$, we find a set of points $\{x_0,x_1,...\}$.

The sequence $x_0,x_1,x_2,...$ is called the \textbf{orbit} starting from $x_0$.

A \textbf{fixed point} of a map $x_{n+1} = f(x_n)$ is a point where $f(x_n) = x_n$

 

\end{tcolorbox}

\vspace{0.2cm}

\hrule
\vspace{0.2cm}

\includegraphics[width=0.9\textwidth]{img/191018-C18p1.png}

To approximate a Poincar\'e map, I start on the line segment $S$ (shown in black in the phase portraits) and integrate forward in time until the trajectory (shown in orange in the phase portraits) next crosses $S$.



\includegraphics[width=1.5in]{img/191018-C18p2.png}



In the graph above, I am plotting the $y$-value at two successive crossings of $S$.  I start at a point $(x_0,y_0)$ on $S$ and return to $S$ at the point $(x_1,y_1) = P(x_0,y_0)$.  I am plotting points $(y_0,y_1)$ in blue.  This is a representation of the Poincar\'e map.  In orange, I am plotting the line where $y_1 = y_0$.  When $y_1 = y_0$, we have a fixed point.
%\vspace{1in}



\vspace{0.2cm}

\hrule
\vspace{0.2cm}

\noindent\textbf{Skill Check C18 practice}
\begin{questions}
\item Retake of skill check C15: trapping region for polar system

\item 

Let the line segment $\Sigma$ be the section of the $x$-axis given by $1\leq x \leq 5$.  A Poincar\'e map taken along a line segment $\Sigma$ is shown on the left (blue curve).  Sketch two trajectories that are consistent with the Poincar\'e map on the axes to the right. 

\includegraphics[width=0.8\textwidth]{img/examp2Lorenz.png} 

\vspace{0.2cm}

\hrule
\vspace{0.2cm}
\end{questions}



\noindent\textbf{Skill check C18 practice solution}


There's a fixed point of $P(x)$ at about $3.4$, so a closed orbit through $(3.4,0)$ is one trajectory consistent with the map.  $x = 5$ has $P(x) \approx 4$ so a trajectory that connects $(5,0)$ to $(4,0)$ (and stays outside the closed orbit) is also consistent with the map.



\vspace{0.2cm}

\hrule
\vspace{0.2cm}

\noindent \ \ 0.  Discuss your ice cream and dessert preferences with the other members of your team (and write your names on the slide).

\begin{questions}

\item (convince yourself of the details: van der Pol example) After a change of variables the van der Pol system is
\begin{align*}
\dot{x} = &\  \mu(y-\frac{1}{3}x^3 + x) \\
\dot{y} = &\ -\frac{1}{\mu} x.
\end{align*}
Consider the case where $\mu>>1$.
\begin{parts}
\item In this relaxation oscillation, the trajectory is moving very quickly when it jumps between the two parts of the $\dot{x}=0$ nullcline.
Using the $\dot{x}$ equation, convince yourself that it is moving at a velocity of $c\mu$, where $c$ is between $0.1$ and $2$ for most of the jump.  

\emph{On the plot below, a single trajectory is shown in red.  The $\dot x = 0$ nullcline is drawn with a solid black line.  The dashed lines are the curves $y - x^3/3 + x = \pm 0.1$ and the dotted lines are the curves $y - x^3/3 + x = \pm 1.4$.}


\includegraphics{img/191016-C17p3.png}

\emph{We sometimes say this velocity is of the order of $\mu$, or is $\mathcal{O}(\mu)$, because it is bounded above by a constant multiple of $\mu$.}

\item The trajectory is traversing a
distance of about $3$ as it jumps.  Combine the approximate distance and the approximate velocity to find the $\mu$-dependence of the time that it spends jumping.
%\item How close does the trajectory need to be to the $\dot{x} = 0$ 
% nullcline for both $\dot{x}$ and $\dot{y}$ to be the same order of magnitude?
\item While moving along the nullcline, the trajectory moves about $1$ unit in $x$ and a bit less than $2$ units in $y$.  It is basically moving along the curve
$y = \frac{1}{3}x^3-x$ (it is not quite on the curve, but it is close to that curve the whole time).  The time it spends 
traversing the curve is a constant multiple of $\mu^k$ for some integer $k$.  


To estimate time (just as we did for the oscillators in chapter 4), we set up an integral of the form $\displaystyle\int_{x_1}^{x_2} \frac{dt}{dx}dx$
or something like this.  This integral shows us how the time depends on $\mu$.  Using  \[\int_{x_1}^{x_2} \frac{dt}{dx}dx\]
doesn't work so well.  It
puts $y-(\frac{1}{3}x^3-x)$ in the denominator (so a dependence on $x$ and $y$, not just $x$).
What is wrong with having a dependence on $x$ and on $y$ in the integral?

\item We could try again with \[\int_{y_1}^{y_2} \frac{dt}{dy}dy.\]  Argue that this leads to a problem, too, and isn't something we can integrate.

\item 
So actually, Steve used \[\int_{x_1}^{x_2} \frac{dt}{dy}\frac{dy}{dx}dx.\]  
(This is an example of persisting until something works, and luckily getting something to work before we run out of options).  Confirm that this expression results in something that is integrable.

\emph{Note that for $\frac{dy}{dx}$ we're thinking of ourselves as, to good approximation, being
stuck on the nullcline, so compute this directly by assuming the trajectory is exactly on the nullcline.}

\item
Use the setup of this final integral to identify $k$.  There is no need to evaluate the integral.  We just want to learn the $\mu$ dependence of the time.  
\item Compare the amount of time spent jumping to the amount of time spend moving along the curve.  (We have the timescales
of these processes, and not the exact amounts of time, so compare the timescales).
\end{parts}


  
\question (weakly nonlinear van der Pol) Let $E(x,\dot x) = \frac{1}{2}x^2+\frac{1}{2}\dot x^2$.
\begin{parts}
\part Find $\frac{dE}{dt}$ for the weakly nonlinear van der Pol oscillator, where $x+\ddot x = -\epsilon \dot x (x^2-1)$.

\emph{Write $\dot E$ in terms of just $x$ and $\dot x$.}

\part We have $\Delta E = \int_0^T \frac{dE}{dt}dt$.  The weakly nonlinear van der Pol is very close to the system $\ddot x + x = 0$.  Two trajectories are shown for each system in the plots below.

\includegraphics[width=3in]{img/191016-C17p1.png}
\includegraphics[width=3in]{img/191016-C17p2.png}

In the $\ddot x + x = 0$ system, the period of cycles is $T = 2\pi$, and $x(t) = A\cos t$ is a solution for any $A$.  In the weakly nonlinear van der Pol system, assume there exists a limit cycle (we know that there is one from Lienard's theorem), and that it is of the form $x(t) = A\cos t$ with period $T\approx 2\pi$.  We want to find the value of $A$ associated with the limit cycle.
\begin{itemize}
    \item Find $\dot x$ assuming $x(t) = A\cos t$.
    \item Substitute $\dot x$ and $x$ into 

$\Delta E \approx -\epsilon \int_0^{2\pi} \dot x^2 (x^2-1) dt$

\item  Using $\displaystyle \frac{1}{2\pi} \int_0^{2\pi} \sin^2 t dt = \frac{1}{2}$ and $\displaystyle \frac{1}{2\pi}\int_0^{2\pi} \cos^2 t\sin^2 t dt = \frac{1}{8}$, find a value of $A$ such that $\Delta E = 0$.
\item Check the $A$ you found against what is happening in the phase portrait above.
\end{itemize}

\end{parts}

\item Consider the two Poincar\'e maps represented below.  
\begin{itemize}
    \item For each map, find the approximate $y$ value associated with a closed trajectory.
    \item If you start close to the closed trajectory, will you approach the closed trajectory or will you move away from it?
    
    \emph{Use the stability of the fixed point to determine the stability of the closed trajectory.}
\end{itemize}

\includegraphics[width=2in]{img/191018-C18p3.png}
\includegraphics[width=2in]{img/191018-C18p4.png}


% \item (Time averaging an oscillation over a single period) Show the time averaging facts that you used above.

% Find $\left<(\sin t)^2\right>$ or $\left<(\cos t)^2(\sin t)^2\right>,$ where $\left< f \right>$ denotes the average of a periodic function $f$ over a single cycle.  For $f(t)$ a function of period $2\pi$, we can compute this average as $\frac{1}{2\pi}\int_0^{2\pi}f(t)dt$.

% I find that a straightforward way to compute this type of quantity is to use $\cos t = \frac{1}{2}(e^{it}+e^{-it})$ and $\sin t = \frac{1}{2i}(e^{it}-e^{-it})$.  

% Note that $\cos nt = \frac{1}{2}(e^{int}+e^{-int}$.  Also note that $\int_0^{2\pi} \cos nt\ dt =0$ for $n=1,2,3,...$.

\end{questions}

\eject

\textbf{Answers:}

\begin{enumerate}
\item
\begin{enumerate}
\item When the trajectory is flowing across, $\dot{y}$ is small: $\vert x \vert < 3$ or so, and $\mu$ is big, so $\vert \frac{x}{\mu} \vert < \frac{3}{\mu}$, which is small
(specifically order of $\frac{1}{\mu}$.  
This means the motion is basically horizontal.  And it is moving at a speed of $\mu (y-x^3/3 + x)$ in the horizontal direction.
Away from the nullcline itself, $y-x^3/3 + x$ is between $0.1$ and $1.4$ for almost the whole distance across, so $\dot x$ will be between $0.1\mu$ and $1.4\mu$ for most of the jump.
\item It jumps across a distance of maybe $3$ as it
moves horizontally.
Since $distance/time = velocity$ the time is $distance/velocity$, so it is proportional to $\frac{1}{\mu}$.  This will be very small and means that it jumps across pretty quickly.
%\item For $\dot{x}$ and $\dot{y}$ to be the same order we need them to both be order $\frac{1}{\mu}$ so we need $(y- x^3/3 + x)$ to be
%order $\frac{1}{\mu^2}$ so that when it is multiplied by $\mu$ it is order $\frac{1}{\mu}$.  So we need to be within order $\frac{1}{\mu^2}$ of the
% nullcline.
\item Along the nullcline, $x$ and $y$ each depend on $t$, but they also have a relationship with each other.  The integral $\int_{x_1}^{x_2} \frac{1}{y+x-x^3/3}dx$ isn't something that we can integrate because $y$ is coupled to $x$ as we move along the nullcline, so we can't treat $y$ as a constant while we integrate with respect to $x$.  Instead, we would need to write $y$ in terms of $x$.
\item This one is $\int_{y_1}^{y_2} -\frac{\mu}{x}dy$ so also mixes together $x$ and $y$.  As $y$ changes, $x$ also changes, so we can't integrate this without knowing how they relate to each other.  Instead, we would need to write $x$ in terms of $y$.
\item On the nullcline, $y = -x+x^3/3$ so $\frac{dy}{dx} = -1 + x^2$.  $\displaystyle \int_{x_1}^{x_2} -\frac{\mu}{x}(x^2-1)dx$.  This is an integral where the integrand doesn't depend on $y$ or $t$ but just on $x$, and it is being integrated $dx$, so this one can be computed as written.
\item $\frac{dt}{dy} = -\mu\frac{1}{x}$ and $\frac{dy}{dx} = x^2 - 1$ on the nullcline.  So $\displaystyle T = \int_{x_1}^{x_2} -\mu \frac{1}{x}(x^2-1) dx = \mu\int_{x_1}^{x_2} - \frac{1}{x}(x^2-1) dx.$
This is $\mu$ multiplied by a constant that doesn't depend on $\mu$.  So this time is proportional to $\mu$ meaning that $k = 1$.
\item The jump is order of $\frac{1}{\mu}$ and the motion along the curve is order of $\mu$.  This means we spend a ton more time on the 
curve compared to doing the jump.
\end{enumerate}

\item \begin{enumerate}
    \item 
 $\dot E = x \dot x + \dot x \ddot x = \dot x(x + \ddot x) = -\epsilon \dot x^2(x^2-1)$
 \item $\dot x = -A \sin t$.  $\Delta E = -\epsilon \int_0^{2\pi}A^2\sin^2 t(A^2\cos^2 t - 1)dt = -\epsilon\left(A^4/8 -A^2/2\right)$.  $A^4/8 - A^2/2 = 0$ so $A^2/4  = 1 \Rightarrow A = 2$.  This looks like it matches the location where the green trajectory and the orange trajectory touch.  If the orange one is moving outward and the green one inward, then there is a limit cycle at about $A \approx 2$.
 \end{enumerate}
% \item $\frac{1}{2}$ and $\frac{1}{8}$.
\end{enumerate}

\end{document}