\documentclass[12pt,letterpaper,noanswers]{exam}
\usepackage[usenames,dvipsnames,svgnames,table]{xcolor}
\usepackage[margin=0.9in]{geometry}
\renewcommand{\familydefault}{\sfdefault}
\usepackage{multicol}
\pagestyle{head}
\definecolor{c03}{HTML}{FFDDDD}
\header{AM 108 Project Slides}{}{Due April 26th at 10am}
\runningheadrule
\headrule
\usepackage{graphicx} % more modern
\usepackage{amsmath} 
\usepackage{amssymb} 
\usepackage{hyperref}
\usepackage{tcolorbox}

\begin{document}
 \pdfpageheight 11in 
  \pdfpagewidth 8.5in
  
\section{Presentation info}
\subsection{Logistics}

You have slides due on Friday Apr 26th at 10am.  Those slides are for the (6 minute) progress reports on Monday April 29th or Wednesday May 1st.  During class on Friday you'll give and receive feedback around the slides, and can begin revising them for the following Monday.  

\subsection{A few tips}
\url{https://hbr.org/2013/06/how-to-give-a-killer-presentation}
\begin{itemize}
\item A few tips from the article: you are telling a story.  (see ``Frame Your Story'')

\emph{You don't have to share everything you've done in this presentation (that's one thing your individual report will be for - to tell me about other things you tried that didn't make the presentation).}

Instead, find an angle through your question and your project work that tells us a story.  You are familiar with the dynamical systems background of your classmates, and it should be story that is accessible to them.
\item Your presentation slides are there to be illustrations.  (see ``Plan the Multimedia'').  Nothing should be written on the slides that you plan to talk about out loud.  Related images that illustrate what you're talking about are great!

\emph{When I practice a talk, I watch for moments when I am describing something for the audience to imagine, or when I am gesturing a lot.  Then I'll add an image to the slides (often hand-drawn) so that I can point to the image at that moment in the talk.  }
\end{itemize}

\subsection{Content}
\begin{itemize}
\item
This is your first time formally introducing your problem to your classmates.  You'll want to convey your question as well as what's exciting, important, or interesting to you about it.  You'll also want to give the class a sense of how you're approaching the problem.  Share some results as well, and next steps.  Think of this as a six minute advertisement for the talk that you'll give on May 11th.

\item
At least one-third of the talk (so a minimum of 2-3 illustrating slides, assuming a rule of thumb of about one slide per minute) should be focused on your own work and your planned next steps (what you will accomplish by May 11th).  Illustrations for this part might include visualizations of data you're working with, bifurcation diagrams or phase portraits you've made, your own versions of figures from a paper or text that you're following, or drawings that help you convey what you've been doing and what you plan to do.

\item
At most two-thirds of the talk (so a maximum of 3-4 illustrating slides, assuming that same slide-a-minute rule of thumb) can be devoted to explicating your question or project purpose and sharing background information or the previous work of others.  

\item If this format doesn't make sense for the story you want to tell through your talk, please check in with me!  Other formats may work, as well.

\item Note: I use a slide-a-minute rule of thumb for myself, but I know that depending on the slide (if it is part of an image build, for example, or if it is a simple image that doesn't take time to take-in, such as a map with a location marked) that less time can also work.  I imagine that (excluding the title slide) 5-10 slides would be appropriate to illustrate this talk.
\end{itemize}



\end{document}