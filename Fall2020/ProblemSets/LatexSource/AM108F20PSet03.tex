\documentclass[12pt,letterpaper,noanswers]{exam}
%\usepackage{color}
\usepackage[usenames,dvipsnames,svgnames,table]{xcolor}
\usepackage[margin=0.9in]{geometry}
\renewcommand{\familydefault}{\sfdefault}
\usepackage{multicol}
\pagestyle{head}
\definecolor{c03}{HTML}{FFDDDD}
\header{AM 108 Problem Set 03}{Updated on \today.}{Due Fri Sept 25\\ at 5pm ET}
\runningheadrule
\headrule
\usepackage{diagbox}
\usepackage{graphicx} % more modern
%\usepackage{subfigure} 
\usepackage{amsmath} 
\usepackage{amssymb} 
%\usepackage{gensymb} 
%\usepackage{natbib}
\usepackage{hyperref}
%\usepackage{enumitem}
%\setlength{\parindent}{0pt}
%\usepackage{setspace}
%\pagestyle{empty}  
%\newcommand{\Sc}[0]{
%{\color{BlueViolet}\S}
%}
\usepackage{tcolorbox}
\usepackage[framed,numbered,autolinebreaks,useliterate]{mcode}

\begin{document}
 \pdfpageheight 11in 
  \pdfpagewidth 8.5in
\begin{itemize}
\item This problem set is due by 5pm on Friday September 25th.  Upload your written work and screenshots of your Mathematica work to Gradescope.  Upload your Mathematica file to Canvas.
\item Fill out the online cover sheet (on Canvas) for each assignment to name your collaborators, list resources you used, and estimate the time you spent on the assignment.
\end{itemize}

\noindent\textbf{Academic Integrity and Collaboration on Problem Sets:}  

Collaborating with classmates in planning and designing solutions to homework problems is encouraged.  Collaboration, cooperation, and consultation can all be productive.  Work with others by 
\begin{multicols}{2}
\begin{itemize}
\itemsep-0.2em
    \item discussing the problem,
    \item brainstorming,
    \item walking through possible strategies,
    \item outlining solution methods
\end{itemize}   
\end{multicols}

\noindent For problem sets, you may consult or use:
\begin{multicols}{2}
\begin{itemize}
\itemsep-0.2em
    \item Course text (including answers in back)
    \item Other books
    \item Internet
    \item Your notes (taken during class)
    \item Class notes of other students
    \item Course handouts
    \item Piazza or Slack posts from the course staff
    \item Computational tools such as Mathematica or Desmos
    \item Calculators
\end{itemize}
\end{multicols}

\noindent You may \textbf{not} consult:
\begin{multicols}{2}
\begin{itemize}
\itemsep-0.2em
    \item Solution manuals
    \item Problem sets from prior years
    \item Solutions to problem sets from prior years
    \item Other sources of solutions
    \item Emails from the course staff
\end{itemize}
\end{multicols}

\noindent You may:
\begin{itemize}
\itemsep-0.2em
    \item Look at communal work while writing up your own solution
    \item Copy computer code from the source files provided with the problem sets
    \item Look at a screenshare of another student's computer code
\end{itemize}

\noindent You may \textbf{not}
\begin{itemize}
\itemsep-0.2em
    \item Look at the individual mathematical work of others
    \item Post about problems online
    \item Copy and paste computer code from another student (or otherwise directly use the code of another student)
\end{itemize}

\eject

{\color{blue}\href{https://hollis.harvard.edu/primo-explore/fulldisplay?docid=01HVD_ALMA512272214860003941&context=L&vid=HVD2&lang=en_US&search_scope=everything&adaptor=Local%20Search%20Engine&tab=everything&query=any,contains,strogatz%20nonlinear&offset=0}{link to book on Hollis}}

\begin{questions}
\question (based on 3.7.5)  Read the problem setup in the text for the framing of this question and for the dynamical system that you're working with.

\begin{parts}
\item Do the nondimensionalization in part (a).  Show your mathematical steps.
\item Analyze the system in the case where $s = 0$, showing your work.  $s = 0$ corresponds to no signal substance, $S$.  Based on your work, what long term behavior(s) would you observe in the system in the absence of signal substance?
\item Create a bifurcation diagram with $s$ along the horizontal axis, and $r$ fixed.  \emph{You may use Mathematica for this; if you do, submit screenshots on Gradescope and code in your source file on Canvas}.  Identify the value of $r$ that you chose.  Name any bifurcation(s) you see in your diagram and mark the bifurcation points along the $s$ axis.
\item Assume $g(0) = 0$, meaning there is no gene product (and $x(0) = 0$).  Suppose $s$ is slowly increased from zero to a large value.  What happens to $g(t)$ (or $x(t)$)?  Suppose $s$ is then slowly decreased back to zero.  Does the gene turn off again (does $g(t)$ return to zero)?
\item How does the behavior you found in (d) depend on the value of $r$ that you chose?  If possible, find a value of $r$ where you would see a different behavior given the scenario in (d).
\item By modifying code in the \texttt{AM108F20PSet03.nb} Mathematica file, create a stability diagram for this system.  In your Mathematica file, include a section label, and text describing the steps you're using to find the stability diagram.  Submit a screenshot of the Mathematica work on Gradescope, and submit your source file on Canvas.
\item Working by hand, find parametric equations for the bifurcation curves in $(r,s)$-space.  Show your steps.  \emph{These are the curves that you plot in your stability diagram}.
\item The title of the problem was ``A biochemical switch'', where the switch refers to a single pulse of some signaling substance turning on a gene (so that the gene then stays on).  According to your analysis, for what range of $r$ does the system have this kind of switch?  How can you see this using your stability diagram?
\end{parts}

\question (4.5.3)  See the text for this question.  For part (a), include a phase portrait drawn on the circle. 

\question (4.5.1) See the text for this question.  For part d: explain what $T_{\text{drift}}$ represents as well as finding the formula.

\question (based on 3.7.6) Read the problem setup in the text for the framing of this question and for the dynamical system that you're working with.
\begin{parts}
\item see the text \emph{One way to do this is to show $\frac{d}{dt}(x+y+z) = 0$, so if $x+y+z$ starts at $N$ it will stay at $N$.}
\item We have $\dot x = -k x y, \dot z = l y$.  Rewrite the $\dot x$ equation using $\dot z$ in place of $y$.  Separate variables, and integrate both sides with respect to time, then manipulate, to find $x(t) = x_0 e^{-kz(t)/l}$, where $x_0 = x(0)$.
\item Write $y$ in terms of $N,x,z$, and use your expression from (b), to show that $z$ satisfies the first-order equation $\dot z = l(N-z-x_0e^{-kz/l})$.
\item Use nondimensionalization to rewrite this equation as $\frac{du}{d\tau} = a - bu - e^{-u}$.  Show your steps, provide expressions for $u,\tau,a,b$ and show that $a\geq 1$ and $b> 0$.
\item Do part (f) in the text.  Read part (g) as well (but no you can skip showing this).
\item Do parts (h) and (i) in the text.  Read part (j) as well (but you can skip writing this out).
\item Do part (k) in the text.
\end{parts}


\end{questions}



\end{document}
