\documentclass[12pt,letterpaper,noanswers]{exam}
%\usepackage{color}
\usepackage[usenames,dvipsnames,svgnames,table]{xcolor}
\usepackage[margin=0.9in]{geometry}
\renewcommand{\familydefault}{\sfdefault}
\usepackage{multicol}
\pagestyle{head}
\definecolor{c03}{HTML}{FFDDDD}
\header{AM 108 Problem Set 06}{Updated on \today.}{Due Fri Oct 16\\ at 5pm ET}
\runningheadrule
\headrule
\usepackage{diagbox}
\usepackage{graphicx} % more modern
%\usepackage{subfigure} 
\usepackage{amsmath} 
\usepackage{amssymb} 
%\usepackage{gensymb} 
%\usepackage{natbib}
\usepackage{hyperref}
%\usepackage{enumitem}
%\setlength{\parindent}{0pt}
%\usepackage{setspace}
%\pagestyle{empty}  
%\newcommand{\Sc}[0]{
%{\color{BlueViolet}\S}
%}
\usepackage{tcolorbox}
\usepackage[framed,numbered,autolinebreaks,useliterate]{mcode}

\begin{document}
 \pdfpageheight 11in 
  \pdfpagewidth 8.5in
\begin{itemize}
\item This problem set is due by 5pm on Friday Oct 16th.  Upload your written work and screenshots of your Mathematica work to Gradescope.  Upload your Mathematica file to Canvas.
\item Fill out the online cover sheet (on Canvas) for each assignment to name your collaborators, list resources you used, and estimate the time you spent on the assignment.
\end{itemize}

\noindent\textbf{Academic Integrity and Collaboration on Problem Sets:}  

Collaborating with classmates in planning and designing solutions to homework problems is encouraged.  Collaboration, cooperation, and consultation can all be productive.  Work with others by 
\begin{multicols}{2}
\begin{itemize}
\itemsep-0.2em
    \item discussing the problem,
    \item brainstorming,
    \item walking through possible strategies,
    \item outlining solution methods
\end{itemize}   
\end{multicols}

\noindent For problem sets, you may consult or use:
\begin{multicols}{2}
\begin{itemize}
\itemsep-0.2em
    \item Course text (including answers in back)
    \item Other books
    \item Internet
    \item Your notes (taken during class)
    \item Class notes of other students
    \item Course handouts
    \item Piazza or Slack posts from the course staff
    \item Computational tools such as Mathematica or Desmos
    \item Calculators
\end{itemize}
\end{multicols}

\noindent You may \textbf{not} consult:
\begin{multicols}{2}
\begin{itemize}
\itemsep-0.2em
    \item Solution manuals
    \item Problem sets from prior years
    \item Solutions to problem sets from prior years
    \item Other sources of solutions
    \item Emails from the course staff
\end{itemize}
\end{multicols}

\noindent You may:
\begin{itemize}
\itemsep-0.2em
    \item Look at communal work while writing up your own solution
    \item Copy computer code from the source files provided with the problem sets
    \item Look at a screenshare of another student's computer code
\end{itemize}

\noindent You may \textbf{not}
\begin{itemize}
\itemsep-0.2em
    \item Look at the individual mathematical work of others
    \item Post about problems online
    \item Copy and paste computer code from another student (or otherwise directly use the code of another student)
\end{itemize}

\eject

{\color{blue}\href{https://hollis.harvard.edu/primo-explore/fulldisplay?docid=01HVD_ALMA512272214860003941&context=L&vid=HVD2&lang=en_US&search_scope=everything&adaptor=Local%20Search%20Engine&tab=everything&query=any,contains,strogatz%20nonlinear&offset=0}{link to book on Hollis}}



\begin{questions}

\question (working with polar, 7.1.5, 7.3.3) 
\begin{parts}
\part (7.1.5) Show that the system $\dot{r} = r(1-r^2), \dot{\theta} = 1$ is equivalent to 
\begin{align*}
\dot{x} &= x - y - x(x^2+y^2) \\
\dot{y} &= x + y - y(x^2+y^2)
\end{align*}
where $x = r\cos\theta$ and $y = r\sin\theta$.  No need to analyze the system.

\emph{Recall the multivariable chain rule: $\frac{dx}{dt} = \frac{dr}{dt}\cos\theta - r\frac{d\theta}{dt}\sin\theta$.}

\emph{You are welcome to use Mathematica for this; in that case explain how you used Mathematica as part of your writeup (and submit screenshots on Gradescope along with your source code on Canvas for completeness).}


\part (7.3.3) Do this problem as written.
% Consider the system $\dot{x} = x - y - x^3, \dot{y} = x + y - y^3$.  Rewrite the system in polar coordinates, where $r^2 = x^2 + y^2, \tan\theta = \frac{y}{x}$.  Then construct a trapping region and use the Poincar\'e-Bendixson Theorem to show that the system has a closed trajectory.

% \emph{From the chain rule: $r\dot{r} = x\dot{x} + y\dot{y}$ and $\dot{\theta} = (x\dot{y} - y\dot{x})/r^2$.}

\emph{You are welcome to use Mathematica to help you; in that case explain how you used Mathematica as part of your writeup, include screenshots, etc.}

\emph{Remember that trapping regions for the Poincar\'e-Bendixson theorem need to be closed sets, so they need to include their boundary points.}



\end{parts}

\question (7.2.9) Do this problem as written.  For (c) there's a typo in the text.  The system should be $\dot x = -2xe^{x^2+y^2}$, $\dot y = -2y e^{x^2+y^2}$.
\question (7.2.12) Show that $\left\{\begin{array}{c} \dot{x} = -x + 2y^3-2y^4 \\ \dot{y} = -x-y+xy\end{array}\right.$ has no periodic solutions.


Try $V(x,y) = x^m + a y^n$, and choose $a, m, n$ so that it is a Liapunov (Lyapunov) function for this system.




\emph{You are welcome to use Mathematica for this; be sure to describe your setup in your writeup and to submit your code on Canvas.}




\question  (6.8.14) Consider the family of linear systems $\left\{\begin{array}{c} \dot{x} = x\cos\alpha - y\sin\alpha \\ \dot{y} = x\sin\alpha + y\cos\alpha\end{array}\right.$, where $\alpha$ is a parameter that runs over the range $0\leq \alpha\leq \pi$.



\begin{parts}
\item Working analytically, classify the fixed point at the origin as a function of $\alpha$.  Generate phase portraits for a few values and include the computational tool you used along with any specific commands as part of your solution.  



%\emph{You can upload images onto Overleaf and can also insert code using the verbatim environment.}
%\emph{Submitting a mathematica file separately is just fine.  Include text in the file explaining what you're doing, and remember to use complete sentences.}
\item \[I_C = \frac{1}{2\pi}\oint_C \frac{f\ dg - g\ df}{f^2+g^2}\] is an integral that gives you $I_C$.  

Use this integral to show that $I_C$ is independent of $\alpha$. %(and see 6.8.13 for the derivation of the integral).

\emph{You may have questions about notation - please post them to the class on Piazza.  $df$ and $dg$ are differentials.  They capture the change in $f$ or in $g$ as either or both $x, y$ changes so $df = f_x dx + f_y dy$.}



\item Let $C$ be a circle centered at the origin.  Compute $I_C$ explicitly by evaluating the integral.


\emph{You may not have worked with line integrals in some time, and may not have used differential notation for them before.  Below is a worked example taken from a calculus textbook:}

Example Q: Evaluate $\int_C x y dx - y^2 dy$ where $C$ is the line segment from $(0,0)$ to $(2,6)$.

Example solution: First, parameterize the curve $C$.  Let $x = t$, so for $x$ to go from $0$ to $2$ we choose $0 \leq t \leq 2$.  Next, find $y$ in terms of $t$ so that at each time $t$ we are at a point along the curve.  Since $y = 3x$ is our curve, let $y = 3t$.  Our parameterization is $x(t) = t, y(t) = 3t, 0\leq t\leq 2$.

Next, find $dx$ and $dy$.  This is just like what you do in a $u$-substitution.  Here, $dx = 1 dt$ and $dy = 3dt$.

Now plug this all in to the integral:  $\int_C xydx - y^2 dy = \int_0^2 (t)(3t)dt - (3t)^23dt$.  Next, simplify and evaluate:  $\int_0^2 (3t^2 - 27t^2)dt = -24 \int_0^2 t^2 dt$.



\end{parts}


\end{questions}



\end{document}
