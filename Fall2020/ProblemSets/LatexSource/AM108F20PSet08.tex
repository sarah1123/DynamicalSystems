\documentclass[12pt,letterpaper,noanswers]{exam}
%\usepackage{color}
\usepackage[usenames,dvipsnames,svgnames,table]{xcolor}
\usepackage[margin=0.9in]{geometry}
\renewcommand{\familydefault}{\sfdefault}
\usepackage{multicol}
\pagestyle{head}
\definecolor{c03}{HTML}{FFDDDD}
\header{AM 108 Problem Set 08}{Updated on \today.}{Due Fri Oct 30\\ at 5pm ET}
\runningheadrule
\headrule
\usepackage{diagbox}
\usepackage{graphicx} % more modern
%\usepackage{subfigure} 
\usepackage{amsmath} 
\usepackage{amssymb} 
%\usepackage{gensymb} 
%\usepackage{natbib}
\usepackage{hyperref}
%\usepackage{enumitem}
%\setlength{\parindent}{0pt}
%\usepackage{setspace}
%\pagestyle{empty}  
%\newcommand{\Sc}[0]{
%{\color{BlueViolet}\S}
%}
\usepackage{tcolorbox}
\usepackage[framed,numbered,autolinebreaks,useliterate]{mcode}

\begin{document}
 \pdfpageheight 11in 
  \pdfpagewidth 8.5in
\begin{itemize}
\item This problem set is due by 5pm on Friday Oct 23rd.  Upload your written work and screenshots of your Mathematica work to Gradescope.  Upload your Mathematica file to Canvas.
\item Fill out the online cover sheet (on Canvas) for each assignment to name your collaborators, list resources you used, and estimate the time you spent on the assignment.
\end{itemize}

\noindent\textbf{Academic Integrity and Collaboration on Problem Sets:}  

Collaborating with classmates in planning and designing solutions to homework problems is encouraged.  Collaboration, cooperation, and consultation can all be productive.  Work with others by 
\begin{multicols}{2}
\begin{itemize}
\itemsep-0.2em
    \item discussing the problem,
    \item brainstorming,
    \item walking through possible strategies,
    \item outlining solution methods
\end{itemize}   
\end{multicols}

\noindent For problem sets, you may consult or use:
\begin{multicols}{2}
\begin{itemize}
\itemsep-0.2em
    \item Course text (including answers in back)
    \item Other books
    \item Internet
    \item Your notes (taken during class)
    \item Class notes of other students
    \item Course handouts
    \item Piazza or Slack posts from the course staff
    \item Computational tools such as Mathematica or Desmos
    \item Calculators
\end{itemize}
\end{multicols}

\noindent You may \textbf{not} consult:
\begin{multicols}{2}
\begin{itemize}
\itemsep-0.2em
    \item Solution manuals
    \item Problem sets from prior years
    \item Solutions to problem sets from prior years
    \item Other sources of solutions
    \item Emails from the course staff
\end{itemize}
\end{multicols}

\noindent You may:
\begin{itemize}
\itemsep-0.2em
    \item Look at communal work while writing up your own solution
    \item Copy computer code from the source files provided with the problem sets
    \item Look at a screenshare of another student's computer code
\end{itemize}

\noindent You may \textbf{not}
\begin{itemize}
\itemsep-0.2em
    \item Look at the individual mathematical work of others
    \item Post about problems online
    \item Copy and paste computer code from another student (or otherwise directly use the code of another student)
\end{itemize}

\eject

{\color{blue}\href{https://hollis.harvard.edu/primo-explore/fulldisplay?docid=01HVD_ALMA512272214860003941&context=L&vid=HVD2&lang=en_US&search_scope=everything&adaptor=Local%20Search%20Engine&tab=everything&query=any,contains,strogatz%20nonlinear&offset=0}{link to book on Hollis}}



\begin{questions}
\question (8.6.9)

We are exploring a model of tree frogs.  We want the model to explain experimental results that are observed with two tree frogs and with three tree frogs.
\begin{parts}
\part With two tree frogs, the observation is that they alternate their croak rhythms to croak a half-cycle apart.  This is called \emph{antiphase synchronization}.

Assume the frogs have identical natural frequencies and the same response function to hearing other frogs.

A model of the interaction is
\begin{align*}
\dot{\theta_1}&=\omega+H(\theta_2-\theta_1) \\
\dot{\theta_2}&=\omega+H(\theta_1-\theta_2),
\end{align*}
where $H$ is the coupling function.  Assume it is odd, smooth, and $2\pi$-periodic.

Rewrite this system in terms of the phase difference $\phi = \theta_1 - \theta_2$. 

\emph{Recall that an odd function is a function where $f(-x) = -f(x)$.}



\item Identify the values of $\phi$ that should be associated with stable fixed points for the model to be consistent with the experimental results.  \emph{Provide your (brief) reasoning}.
% To explain the experimental results you will want $\phi \rightarrow \pi$ or $-\pi$ at $t\rightarrow\infty$ (so the stable fixed points should be $\phi = \pm \pi$).





\part Show that the experimental results for two frogs are consistent with the simple interaction function $H(x) = a\sin x$, if the sign of $a$ is chosen appropriately.


\part With three of the frogs interacting, they cannot each be half a cycle away from the others.  They have been observed to settle into one of two distinctive patterns.  One stable pattern involves a pair calling in unison with the third half a cycle out of phase of both.  The other stable pattern has the three frogs maximally out of sync, with each calling one-third of a cycle apart from the other two.

A model of the interaction for three frogs is
\begin{align*}
\dot{\theta_1}&=\omega+H(\theta_2-\theta_1) +H(\theta_3-\theta_1) \\
\dot{\theta_2}&=\omega+H(\theta_3-\theta_2)+H(\theta_1-\theta_2) \\
\dot{\theta_3}&=\omega+H(\theta_1-\theta_3)+H(\theta_2-\theta_3). \\
\end{align*}
The interaction function needs to be the same as for the two-frog model (since it is the same frogs interacting).

Rewrite this system in terms of the phase differences $\phi = \theta_1 - \theta_2$ and $\psi = \theta_2 - \theta_3$.


\item Identify the values of $\phi$ and $\psi$ that should be associated with stable fixed points for the model to be consistent with the experimental results.

%To explain the experimental results you would want $\phi \rightarrow 0$ with $\psi \rightarrow \pi$ or you would want $\phi \rightarrow \pi$ with $\psi \rightarrow 0$ or you would want $\phi \rightarrow \pi$ with $\psi \rightarrow\pi$ or you would want $\phi \rightarrow 2\pi/3$ and $\psi \rightarrow 2\pi/3$ as $t\rightarrow \infty$ (or both to $4\pi/3$).




\part Show that $H(x) = a \sin x$ cannot account for the three-frog results.

\emph{You can use Mathematica for your analysis of the fixed points.  Submit your *.nb file and include screenshots of your work on Gradescope.}


\end{parts}

\question For 8.2.6 and 8.2.7, use numerical methods to determine whether the Hopf at $\mu = 0$ appears to be supercritical or subcritical.

\emph{Identifying whether 8.2.5 is subcritical or supercritical is worked as an example in \texttt{AM108F20PSet08.nb}.}

Use a Poincar\'e map as part of your work.

\question Read 8.1.15.  \textbf{There is a typo in the equations in the text for 8.1.15}.  Use
\begin{align*}
\dot n_A &= (p+n_A)n_{AB} - n_An_B \\
\dot n_B &= n_B n_{AB} -(p+n_A)n_B
\end{align*}
rather than the equations in the text.
\begin{parts}
\item Do part (a) as written.
\item Create two streamplots, one on either side of $p=0.134$ (choose $p$ close to $0.134$).  Add the trajectory described in the text (initial conditions of everyone believing in $B$ except the true believers) to each streamplot.
\item Use Mathematica to find all of the fixed points.  Plot the $x$ values vs $p$ for each fixed point.  
\begin{verbatim}
fp = Solve[f[x, y] == 0, {x, y}]
Plot[Evaluate[x /. fp], {p, 0, 1}]
\end{verbatim}
(Using "Evaluate" will make the four curves different colors).
\item Use Mathematica to find the trace and determinant for each fixed point:
\begin{verbatim}
jacobian = Grad[f[x, y], {x, y}];
tr[p_] = Tr[jacobian] /. fp
det[p_] = Det[jacobian] /. fp
Plot[Evaluate[tr[p]], {p, 0, 1}, PlotRange -> All]
Plot[Evaluate[det[p]], {p, 0, 1}, PlotRange -> All]
\end{verbatim}
\item Use the trace and determinant plots to identify the stable fixed points (there are two).  What type of bifurcation occurs at $p\approx 0.134$?  What kinds of fixed points are involved?
\item For $p<0.134$, and $n_A = 0, n_B = 1-p$, describe the beliefs of people in this system after a long time (according to this model). For $p>0.134$ and $n_A = 0, n_B = 1-p$, describe the beliefs after a long time.
\end{parts}

\end{questions}



\end{document}
