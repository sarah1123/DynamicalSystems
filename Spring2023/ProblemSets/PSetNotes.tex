\documentclass[12pt,letterpaper]{exam}
%\usepackage{color}
\usepackage[usenames,dvipsnames,svgnames,table]{xcolor}
\usepackage[margin=0.9in]{geometry}
\renewcommand{\familydefault}{\sfdefault}
\usepackage{multicol}
\pagestyle{head}
\header{AM 108 Problem Set Notes 2023}{Updated on \today.}{}
\runningheadrule
\headrule
\usepackage{diagbox}
\usepackage{graphicx} % more modern
%\usepackage{subfigure} 
\usepackage{amsmath} 
\usepackage{amssymb} 
\usepackage{hyperref}
\usepackage{tcolorbox}
\usepackage[framed,numbered,autolinebreaks,useliterate]{mcode}

\begin{document}
 \pdfpageheight 11in 
  \pdfpagewidth 8.5in
\section{PSet01}
\begin{enumerate}
\itemsep0pt
\item ID fixed points from a sketch, ID stability from sketch, approximate functions, linear stability analysis, solution sketches, plots in Mathematica
\item more practice with 1
\item find fixed points as function of parameter, compute bifurcation values by hand, create a bifurcation diagram (including stability)
\item consider different parameter cases, recognize bifurcations as a parameter changes, more practice with 3
\end{enumerate}

Class 01-04:
add an example where they see graphs on either side of a saddle-node bifurcation (as a line moves).  make sure there are three fixed points on one side and one on the other.

1d fixed points, stability, phase portraits, linear stability, saddle-node, pitchfork.

\end{document}